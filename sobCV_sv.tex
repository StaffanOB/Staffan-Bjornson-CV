\documentclass{sobCV}[2015/09/08]
\usepackage{tabularx}

\begin{document}
    % =============================================================================
    % Personal information 
    \longcontact{Staffan Björnson}
       {Kornvägen 9}
       {179 63 Stenhamra}
       {2025-03-09}
       {+46 705 86 94 08} 
       {staffan.bjornson@toor.se}


    \banner{sob_banner.png}
    \name{Staffan}{Björnson}


   % =============================================================================
   % A quote that defines you
   \userquote{
       Jag föredrar att göra saker ordentligt från början, istället för att 
       stressa och behöva göra om det.
   }


   % =============================================================================
   % About
   \userAbout{Evangelist inom testautomation och automationsflöden.} 
   {
        Jag är en testautomatiserare med gedigen erfarenhet av att bygga och integrera testautomation i agila utvecklingsflöden. Med en bred bakgrund inom både mjukvaru- och hårdvarutestning har jag en bevisad förmåga att identifiera optimala scenarier för när automation är mest effektivt och när manuella tester är mer lämpliga.

        Under min karriär har jag framgångsrikt utvecklat och implementerat robusta automationslösningar som effektiviserat CI/CD-pipelines och förbättrat den övergripande produktkvaliteten. Mitt arbete har lett till betydande tidsbesparingar i testcykeln, bättre felupptäckt och mer effektiva testprocesser.

        Jag är känd för mitt fokus på långsiktiga lösningar och kvalitet. Jag levererar konsekvent dokumentation av hög kvalitet, delar med mig av min kunskap över teamgränser och implementerar hållbara lösningar.
   }{
        Test och verifiering,
        Testledning,
        Testautomation,
        Hantering av testdata,
        Agilt arbetsflöde,
        CI/CD,Linux/*nix,Robot Framework,
        HSM (Hardware Security Module),
        Verifiering av POS-terminaler
    }


% =============================================================================
% Your job experience 
% =============================================================================
\section{Uppdrag i fokus}
    \experiencefocus{Testautomationsspecialist}{Tele2}
    Som testautomationsspecialist implementerade och drev jag initiativ inom testautomation och CI/CD i samarbete med över 20 team. Jag tog fram en teststrategi för de olika utvecklingsteamen inom Tele2 för att underlätta och förbättra testprocessen, med fokus på integration i CI/CD-flöden och testautomation.


\section{Erfarenhet}
   \experiencenode
   {Testspecialist / Testautomatiserare}
   {Åklagarmyndigheten}
   {Nov 2023 -- Nov 2024}
   {Stockholm, Sverige}{
       Skapa en teststrategi och hjälpa till med implementeringen av testautomation för avdelningen och ett nytt SCRUM-team.
   }
   {Arbetsuppgifter}{
       Testanalys,
       Testimplementering,
       Utveckling av teststrategi,
       Utveckling av agil strategi,
       Implementering av E2E-tester,
       Utbildning i testteori
 }{
     \jobDetails{Testautomation}
      {
       Jag har genomfört en analys av hur Åklagarmyndigheten arbetar med test idag och presenterat ett förslag på hur de kan arbeta mer effektivt med testning i linje med ett agilt arbetssätt.

       Jag har även uppdaterat befintliga E2E-tester med fokus på BAU-flöden (Business As Usual) så att testerna är i linje med hur applikationen och teamet arbetar idag.
     }
 }{
     Testautomation,
     Hantering av testdata,
     Test och verifiering,
     Agilt arbetsflöde,
     SCRUM,
     Windows,
     React,
     SVN,
     Utbildning i API-testning
}
\experiencenode
   {Testspecialist / Testautomatiserare}
   {PwC}
   {Nov 2022 -- Nov 2023}
   {Stockholm, Sverige}{
       Utvecklade och implementerade ett automatiserat testframework för en kundspecifik applikation som kördes i en Azure DevOps-miljö.
   }
   {Arbetsuppgifter}{
       Skapa automatiserade API-tester, 
       Hantera manuella testsuiter,
       Samarbeta med utvecklare för att ta fram enhetstester,
       Utveckla en agil teststrategi,
       Driva det agila arbetssättet
 }{
     \jobDetails{Testautomation}
      {
         Skapade ett automatiserat testframework och skrev om manuella tester till automatiserade, i nära samarbete med utvecklare, kravanalytiker (BA) och produktägare (PO). Introducerade Robot Framework i kombination med Gherkins "Given-When-Then"-format för att möjliggöra en heltäckande automationslösning.

         Tekniskt ansvar för utveckling av automatiserade tester, främst med Robot Framework och Python. Omvandlade manuella testsuiter, rensade bort gamla eller ineffektiva tester och identifierade möjliga vägar framåt för automation.
     }
 }{
       Testautomation,
       Hantering av testdata,
       Agilt arbetsflöde, 
       Windows, 
       Azure DevOps, 
       React, 
       Python, 
       Gherkin, 
       Selenium, 
       Robot Framework,
       Git, 
       Docker, 
       API-testning 
}

\experiencenode
   {Testautomationsspecialist}
   {Tele2 AB}
   {2019 -- Dec 2022}
   {Stockholm, Sverige}{
       Arbetade som testautomationsspecialist inom kvalitetsförbättring och kravinsamling. Jag implementerade och drev initiativ inom testautomation och CI/CD i samarbete med utvecklingsteamen.
   }
   {Arbetsuppgifter}{
      Implementering av CI/CD,
      Utveckling av teststrategi, 
      Införande av agila arbetsflöden,
      Utbildning i agil metodik,
      Omvandling från manuella till automatiserade tester
   }{
       \jobDetails{Testautomationsspecialist} {
           Som medlem i CI/CD-teamet var mitt uppdrag att utvärdera utvecklingsteamen för att kunna automatisera och bli CI/CD-kompatibla. Jag skapade specifika teststrategier för olika utvecklingsteam inom Tele2 för att underlätta och förbättra testprocessen.
       }

       \jobDetails{Testautomation} {
           En viktig del av uppdraget var att identifiera utvecklingsteamens behov och hitta ett ramverk som fungerar för dem vid automatisering. Uppdraget inkluderade även att hålla utbildningar i vad man bör tänka på vid införande av automation och vanliga fallgropar.
       }

       \jobDetails{Kravanalys / Testledning} {
           Jag identifierade brister i kvalitetsprocessen och skapade en övergripande teststrategi för Tele2 som implementerades. En del av detta arbete var att hålla QA-forum om testautomation för framför allt QA, PO och andra intressenter på Tele2 där vi gick igenom nya innovationer, hur man bör tänka, vad som är "best practice" med mera.
       }
   }{
       Testautomation,
       Hantering av testdata,
       Agilt arbetsflöde, 
       CI/CD, 
       Linux/*nix, 
       Windows, 
       Azure DevOps, 
       React, 
       Python, 
       Gherkin, 
       Selenium, 
       Robot Framework,
       RPA,
       Cypress, 
       Git, 
       Docker, 
       Kubernetes, 
       Kibana, 
       Jira, 
       Confluence, 
       TestRail, 
       API-testning 
   }

   
   

\experiencenode{Teknisk testare}
   {Kambi Group}
   {2018 -- Okt 2019}
   {Uppsala, Sverige}{
       Testledare i ett autonomt agilt team som levererade oddskalkyleringsfunktioner för sportspel.
   }
   {Arbetsuppgifter}{
       Testledning,
       Kravarbete i samarbete med intressenter,
       Manuell systemtestning / utforskande testning,
       Skapande av automatiserade tester (UAT och regression)
 }{
     \jobDetails{Testare}{
         Som testare i ett agilt team skapade jag en teststrategi och genomförde manuella tester med fokus på utforskande testning. Jag initierade även ett teamprojekt för att automatisera våra regressionstester, baserat på Python och Robot Framework, vilket integrerades med vårt deploy-pipeline.
     }
 }{
       Testautomation,
       Hantering av testdata,
       Test och verifiering,
       Agilt arbetsflöde, 
       Linux/*nix, 
       Java, 
       Kotlin, 
       React, 
       Python, 
       Gherkin, 
       Selenium, 
       Robot Framework,
       Cypress, 
       Git, 
       Docker, 
       Kibana, 
       ELK stack, 
       Grafana, 
       Jira, 
       Confluence, 
       API-testning, 
       Maven 
}
\experiencenode{Testledare / Juniorutvecklare}
   {Espland AB}
   {2015 -- Nov 2018}
   {Uppsala, Sverige}{
       Som testare och utvecklare för ett TMS (POS-Terminal Management System) ansvarade jag för testning och efterlevnad av PCI DSS-standarder.
   }
   {Arbetsuppgifter}{
       Mjukvarutestning,
       Hårdvarutestning,
       Regelefterlevnad, 
       Javautveckling,
       Support och försäljning
 }{
     \jobDetails{Teknisk testare}
      {
          Rollen omfattade både arbete som testare och junior utvecklare inom Java. I min roll som testledare hade jag huvudansvar för att säkerställa att tester av egenutvecklad mjukvara inklusive integrationer mot standardsystem genomfördes – både genom automatiserade och manuella tester, där jag deltog aktivt i arbetet.

          Vid leverans av mjukvaruuppdateringar till kund deltog jag även i deras testarbete genom support och direkt arbete i kundens testmiljö.

          Vid behov fungerade jag som förstärkning i CAPSAB-teamet (Card and Payment Solution) med verifiering och testning av betalterminaler enligt både svenska och internationella krav.
     }
 }{
       Testautomation,
       Hantering av testdata,
       Test och verifiering,
       Agilt arbetsflöde, 
       Linux/*nix, 
       Windows, 
       Java, 
       React, 
       Python, 
       Gherkin, 
       Selenium, 
       Robot Framework,
       Cypress, 
       Git, 
       Docker, 
       Kubernetes, 
       Kibana, 
       ELK stack, 
       Grafana, 
       Jira, 
       Confluence, 
       TestRail, 
       Elastic Search, 
       API-testning, 
       Maven 
}

\experiencenode{Teknisk testare / Testmiljöingenjör}
   {Verifone}
   {2011 -- 2014}
   {Uppsala, Sverige}{
       Som mjukvaru- och hårdvarutestare utförde jag manuella tester och byggde även testmiljöer för våra team med integration mot HSM och tredjepartssystem.
   }
   {Arbetsuppgifter}{
       Mjukvarutestning,
       Hårdvarutestning,
       Regelefterlevnad, 
       POS-terminaler,
       Hantering av testmiljöer,
       Agila arbetsflöden,
       Support och försäljning
 }{
     \jobDetails{Teknisk testare}
      {
          Jag genomförde manuella acceptanstester i nära samarbete med utvecklarna (parprogrammering enligt XP). Vi var det första teamet på Verifone som arbetade enligt SCRUM, vilket innebar mycket arbete med att hitta ett effektivt och korrekt agilt arbetssätt.

          På Verifone ansvarade jag även för den interna testmiljön – från operativsystemsnivå till testverktyg och integrationer med HSM. Jag hjälpte bland annat till att bygga ett system för automatisk OS- och mjukvarudistribution till testmiljöer.
     }
 }{
       Hantering av testdata,
       Test och verifiering,
       Agilt arbetsflöde, 
       Linux/*nix, 
       Windows, 
       Java, 
       Vaadin, 
       Python, 
       Gherkin, 
       Selenium, 
       Robot Framework,
       Cypress, 
       Git, 
       Docker, 
       Kubernetes, 
       Kibana, 
       ELK stack, 
       Grafana, 
       Jira, 
       Confluence, 
       TestRail, 
       API-testning, 
       Maven 
}

% =============================================================================
% Your Education
% =============================================================================
\section{Kurser}
\coursenode{2024}{Intensivkurs för testledare}{
    En kurs som syftar till att ge kunskap om hur en testledare ska planera, förbereda, genomföra och avsluta ett testprojekt. Kursen är perfekt för den som har ansvar för ett testprojekt eller ett testteam.
}

\coursenode{2023}{DevOps 2023 Learn CI/CD Process for Professionals and Teams}{
    DevOps och CI/CD är ett tankesätt. Kursen behandlar CI/CD-praktiker för agila team, inklusive tips för distansarbete.
}

\coursenode{2023}{Elegant Automation Frameworks with Python and pyTest}{
    Bygg ett snabbt automationsramverk med minimal komplexitet och tydlig rapportering med hjälp av pyTest och Python.
}

\coursenode{2020}{Linux Security and Hardening, The Practical Security Guide}{
    Säkerhetsprinciper och riktlinjer för att stärka säkerheten i Linux-system.
}

\coursenode{2018, 2019}{Robot Framework Test Automation (Level 1-2 \& Jenkins CI)}{
    Automatisera komplexa interaktioner mellan flera webbapplikationer. Integrera Robot Framework-tester i en Jenkins CI-pipeline.
}

\newpage
% =============================================================================
% Skills
% =============================================================================
\section{Färdigheter}
\subsection{Tekniska}
\skills{3}{
    Robot Framework,
    pyTest,
    Selenium,
    Linux,
    Linux-säkerhet,
    Windows,
    Azure DevOps,
    Analytisk förmåga,
    Testautomation,
    Testplanering,
    Agila metoder,
    Agil testning,
    Scrum,
    Systemtestning,
    Python,
    Jenkins,
    Red Hat Satellite,
    FreeIPA,
    Betalningslösningar,
    Selenium 
}

% =============================================================================
% Skills - What languages do you speak and write, separate with a comma.
\subsection{Språk}
\spokenLanguage{
    Svenska,
    Engelska
}
% =============================================================================
% Skills - Do you have a driver's license, separate with a comma.
\subsection{Körkort}
\license{
    A – Motorcykel,
    B – Personbil
}

% =============================================================================
% Social
  \section{Socialt}
     \sosialLink{LinkedIn}{www.linkedin.com/in/staffanob/}{linkedin}
     \sosialLink{GitHub}{github.com/StaffanOB/}{github.png}
     \sosialLink{Toor.se}{www.toor.se/}{homepage}



      

\end{document}
